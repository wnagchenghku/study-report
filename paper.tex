% \documentclass[10pt]{sosp2015}
% \documentclass[10pt,onecolumn]{sosp2015}
\documentclass[10pt,onecolumn,letterpaper]{article}
% \usepackage[10pt,inchmargins]{sigmin}  %% template from Xi Wang.
%\special{papersize=8.5in,11in}
%\setlength{\pdfpagewidth}{8.5in}
%\setlength{\pdfpageheight}{11in}
\usepackage[noheadfoot,
	      left=1in,right=1in,top=1in,bottom=1in,
             columnsep=0.3in
             ]{geometry}
\usepackage[small,compact]{titlesec}
\usepackage[font={small,bf}]{caption}    % added 9/10/13
\usepackage[nolineno,noindent,norules]{lgrind}
\usepackage{tightenum}
\usepackage{float}
\usepackage{xspace}
\usepackage{times,pifont}
\usepackage{mathptmx}
\usepackage{subfig,graphics,graphicx,color}
\usepackage{multirow}
\usepackage{dblfloatfix} %% correctly orders single- and double-col figures
\usepackage{hyphenat}
\usepackage{mathrsfs}
\usepackage{subfig}
\usepackage{amssymb,amsmath,centernot}
\usepackage{lastpage}
\usepackage{flushend}
\usepackage{hhline}
\usepackage{authblk}
\usepackage{booktabs}
%\newcommand{\doi}{XXXXXX}


%%% ================= START of SOSP '13 template ================= 
% \makeatletter
% 
% \def\ftype@copyrightbox{8}
% \def\@copyrightspace{
% \@float{copyrightbox}[b]
% \begin{center}
% \setlength{\unitlength}{1pc}
% \begin{picture}(20,6.0) 
% \put(0,3){\parbox{\columnwidth}{\scriptsize
% 
% %*** SAMPLE. AUTHOR PUT SUPPLIED TEXT HERE ****
% 
% \noindent
% \rule{6.0 cm}{0.2pt}\\
% Permissiondddd to make digital or hard copies of part or all of this work 
% for personal or classroom use is granted without fee provided that
% copies are not made or distributed for profit or commercial advantage 
% and that copies bear this notice and the full citation on the first
% page. Copyrights for third-party components of this work must be
% honored.  For all other uses, contact the Owner/Author. 
% 
% \vspace{\baselineskip}\noindent
% Copyright is held by the Owner/Author(s).\\
% \textit{SOSP'15}.\\
% ACM XXXXXXX.
% 
% \noindent
% http://dx.doi.org/\doi}
% }
% \end{picture}
% \end{center}
% \end@float}
% 
% \def\maketitle{\par
%  \begingroup
%    \def\thefootnote{\fnsymbol{footnote}}
%    \def\@makefnmark{\hbox
%        to 0pt{$^{\@thefnmark}$\hss}}
%      \twocolumn[\@maketitle]
% \@thanks
%  \endgroup
%  \setcounter{footnote}{0}
%  \let\maketitle\relax
%  \let\@maketitle\relax
%  \gdef\@thanks{}\gdef\@author{}\gdef\@title{}\gdef\@subtitle{}\let\thanks\relax
%  \@copyrightspace}
% 
% \makeatother

%%% ================= END of SOSP '13 template ================= 



%\newcommand{\comment}[1]{}
\frenchspacing

%\doublespacing

%%%%%%%%%%%%%%%%%%%%%%%%%%%%
%     macro

\newcommand{\xxx}[0]{\textsc{APUS}\xspace}
\newcommand{\paxos}[0]{\textsc{Paxos}\xspace}
\newcommand{\mytitle}[0]{\textbf {Study Report}}
\newcommand{\mykeywords}[0]{State Machine Replication, Fault Tolerance, Stable 
and Deterministic Multithreading, Software Reliability}

%%%%%%%%%%%%%%%%%%%%%%%%%%%%%%%%%%%%%%%%%%%%%%%%%%%%%%%%%%%%%%%%%
% hyperref stuff

%\usepackage[square,comma,numbers,sort]{natbib}
\usepackage{hypernat}
\usepackage{hyperref}

%% fill in pdf info here
\hypersetup{%
colorlinks=false,
pdfborder={0 0 0},
pdftitle={\mytitle},
pdfkeywords={\mykeywords},
bookmarksnumbered,
pdfstartview={FitH},
urlcolor=cyan,
pdfpagelabels=true,
pdfdisplaydoctitle=true,
}%

%\usepackage{breakurl}
%\usepackage[all]{hypcap}
%\renewcommand{\url}{\burl}

%%%%%%%%%%%%%%%%%%%%%%%%%%%%%%%%%%%%%%%%%%%%%%%%%%%%%%%%%%%%
% Some NICE fonts

\newfont{\BIG}{cminch}                             %--- One-inch font
\newfont{\sfbHuge}{cmssbx10 scaled\magstep5}       %-- 25pt sans serif bold
\newfont{\sfbLarger}{cmssbx10 scaled\magstep3}   %-- 12+pt sans serif boldd
\newfont{\sfblarger}{cmssbx10 scaled\magstep2}   %-- 12+pt sans serif bold
\newfont{\sfblarge}{cmssbx10 scaled\magstep1}      %-- 12pt sans serif bold
\newfont{\sfbeleven}{cmssbx10 scaled\magstephalf}  %-- 11pt sans serif bold
\newfont{\sfb}{cmssbx10}                           %-- 10pt sans serif bold
\newfont{\sfeight}{cmss8}                          %-- 8pt sans serif

%%%%%%%%%%%%%%%%%%%%%%%%
%    space tweaking

%\textwidth = 6.5 in
%\textheight = 9.0 in
%\setlength{\topmargin}{-.5in}

%\headheight = 0.0 in
%\headsep = 0.0 in
%\parskip = 0.2in
%\parindent = 0.0in

\renewcommand{\topfraction}{0.95}
\addtolength{\textfloatsep}{-0.1in}
%\addtolength{\floatsep}{0.025in}
\renewcommand\floatpagefraction{.9}
%\renewcommand\bottomfraction{.9}
\renewcommand\textfraction{.1}

\setlength{\parindent}{9pt}

% Rescue
\makeatletter
\def\v#1{{\mbox{\fontfamily{cmtt}\fontsize{\f@size}{\f@size}\selectfont #1}}}

\newcommand{\dmt}[0]{DMT\xspace}
\newcommand{\smt}[0]{StableMT\xspace}
\newcommand{\smr}[0]{SMR\xspace}

\newcommand{\racepro}[0]{\textsc{RacePro}\xspace}
\newcommand{\criu}[0]{\textsc{CRIU}\xspace}
\newcommand{\lxc}[0]{\textsc{LXC}\xspace}
\newcommand{\tern}[0]{\textsc{Tern}\xspace}
\newcommand{\peregrine}[0]{\textsc{Peregrine}\xspace}
\newcommand{\parrot}[0]{\textsc{Parrot}\xspace}
\newcommand{\repframe}[0]{\textsc{RepFrame}\xspace}
\newcommand{\grace}[0]{Grace\xspace}
\newcommand{\coredet}[0]{\textsc{CoreDet}\xspace}
\newcommand{\kendo}[0]{Kendo\xspace}
\newcommand{\dthreads}[0]{\textsc{DThreads}\xspace}
\newcommand{\determinator}[0]{Determinator\xspace}
\newcommand{\dos}[0]{dOS\xspace}
\newcommand{\ddos}[0]{DDOS\xspace}
\newcommand{\timealgo}[0]{time bubbling\xspace}
\newcommand{\ldpreload}[0]{LD\_PRELOAD\xspace}
\newcommand{\ntimeout}[0]{$W_{timeout}$\xspace}
\newcommand{\nclock}[0]{$N_{clock}$\xspace}

\newcommand{\apache}{\v{Apache}\xspace}
\newcommand{\mongoose}[0]{\v{Mongoose}\xspace}
\newcommand{\ab}{\v{ApacheBench}\xspace}
\newcommand{\clamav}{\v{ClamAV}\xspace}
\newcommand{\upnp}{uPnP\xspace}
\newcommand{\mediatomb}{\v{MediaTomb}\xspace}
\newcommand{\mencoder}{\v{mencoder}\xspace}
\newcommand{\mongodb}{\v{MongoDB}\xspace}
\newcommand{\ssdb}{\v{SSDB}\xspace}
\newcommand{\mysql}{\v{MySQL}\xspace}
\newcommand{\sysbench}{\v{SysBench}\xspace}
\newcommand{\zookeeper}{ZooKeeper\xspace}
\newcommand{\libpaxos}{libPaxos\xspace}
\newcommand{\spaxos}{S-Paxos\xspace}
\newcommand{\crane}{\textsc{Crane}\xspace}


\newcommand{\aget}[0]{\v{aget}\xspace}
\newcommand{\pthread}[0]{\mbox{Pthreads}\xspace}
\newcommand{\openldap}[0]{{OpenLDAP}\xspace}
\newcommand{\redis}[0]{{Redis}\xspace}
\newcommand{\bdb}[0]{{Berkeley DB}\xspace}
\newcommand{\vtune}[0]{\v{VTune}\xspace}
\newcommand{\http}[0]{\mbox{HTTP}\xspace}

% In short.
\newcommand{\eg}{{e.g.}}
\newcommand{\ie}{{i.e.}}
\newcommand{\etc}{{etc}}
\newcommand{\para}[1]{\vspace{.00in}\noindent{\bf #1}}
\newcommand{\wrt}{{w.r.t. }}
\newcommand{\cf}{{cf. }}

% Synch and network operations.
\newcommand{\checktimebubble}[0]{\v{check\_add\_timebubble()}\xspace}
\newcommand{\mutexlock}[0]{\v{pthread\_mutex\_lock()}\xspace}
\newcommand{\connect}[0]{\v{connect()}\xspace}
\newcommand{\send}[0]{\v{send()}\xspace}
\newcommand{\sendto}[0]{\v{sendto()}\xspace}
\newcommand{\sendmsg}[0]{\v{sendmsg()}\xspace}
\newcommand{\mywrite}[0]{\v{write()}\xspace}
\newcommand{\pwrite}[0]{\v{pwrite()}\xspace}
\newcommand{\close}[0]{\v{close()}\xspace}
\newcommand{\recv}[0]{\v{recv()}\xspace}
\newcommand{\select}[0]{\v{select()}\xspace}
\newcommand{\poll}[0]{\v{poll()}\xspace}
\newcommand{\epollwait}[0]{\v{epoll\_wait()}\xspace}
\newcommand{\accept}[0]{\v{accept()}\xspace}

% Parrot primitives.
\newcommand{\getturn}[0]{\v{get\_turn()}\xspace}
\newcommand{\putturn}[0]{\v{put\_turn()}\xspace}
\newcommand{\wait}[0]{\v{wait()}\xspace}
\newcommand{\signal}[0]{\v{signal()}\xspace}

% Evaluation stats.
\newcommand{\github}[0]{\url{github.com/columbia/crane}}
% \newcommand{\ntype}[0]{four\xspace}
\newcommand{\nprog}[0]{five\xspace}
\newcommand{\overheadmax}[0]{22.99\%\xspace}
\newcommand{\overhead}[0]{34.19\%\xspace}
\newcommand{\dmtspeedup}[0]{10.5\%\xspace}
\newcommand{\proxyoverhead}[0]{2.33\%\xspace}
\newcommand{\timebubblelow}[0]{66.65\%\xspace}
\newcommand{\timebubblehigh}[0]{93.88\%\xspace}
\newcommand{\recovertime}[0]{1.97ms\xspace}
\newcommand{\downgradetime}[0]{0.36s\xspace}
\newcommand{\mencoderspeedup}{\v{49\%}\xspace}

\newcommand{\us}[0]{\(\mu\text{s}\)\xspace}

\def\LGfsize{\footnotesize}
%\pagestyle{empty}


% \conferenceinfo{SOSP'15}{October 4--7, 2015, Monterey, CA}
% \copyrightyear{2015} 
% \copyrightdata{978-1-4503-3834-9/15/10} 
% \doi{2815400.2815427}

\title{\mytitle}
% \authorinfo{Authors}

% \author[+*]{Heming Cui}
% \author[*]{Rui Gu}
% \author[*]{Cheng Liu}
% \author[x]{Tianyu Chen}
% \author[*]{Junfeng Yang}
% \setlength{\affilsep}{0.5em}
% \renewcommand\AB@affilsepx{\hspace{28.0 mm}\protect\Affilfont}
% \affil[+]{\textrm\fontsize{10}{10}\selectfont The University of Hong Kong}
% \affil[*]{\textrm Columbia University}
% \affil[x]{\textrm Tsinghua University\vspace{-7.0 mm}}

\begin{document}

% Hack for: Package caption Error: No float type 'copyrightbox' defined.
%\newcounter{copyrightbox}

\date{}

\author[+]{}
\maketitle
%\thispagestyle{empty}

% \begin{sloppypar}
% \begin{abstract}
% \input{abstract}
% \end{abstract}
% \end{sloppypar}

% \begin{sloppypar}
%% %\category{D.2.5}{Software Engineering}{Testing and Debugging}
%% \category{D.4.5}{Operating Systems}{Threads, Reliability}
%% \category{D.2.4}{Software Engineering}{Software/Program Verification}
%% \terms{Algorithms, Design, Reliability, Performance}
%% \keywords{\mykeywords}

%% \vskip 2mm
%% \noindent {\small \bf Categories and Subject Descriptors:} \vskip -.2mm
%% \noindent
%% {\footnotesize D.4.5~[{\bf Operating Systems}]: {Threads, Reliability}\\
%% D.2.4~[{\bf Software Engineering}]: {Software/Program Verification};}
%% \vskip 1mm
%% \noindent {\small \bf General Terms:} \vskip -.2mm
%% \noindent
%% {\footnotesize Algorithms, Design, Reliability, Performance}
%% \vskip 1mm
%% \noindent {\small \bf Keywords:} \vskip -.2mm
%% \noindent
%% {\footnotesize \mykeywords}

% \vskip 2mm
% \noindent {\small \bf Categories and Subject Descriptors:}
% {\small D.4.5~[{\bf Operating Systems}]: {Threads, Reliability};
%   D.2.4~[{\bf Software Engineering}]: {Software/Program Verification};}
% \vskip .1mm
% \noindent {\small \bf General Terms:} {\small Algorithms, Design,
%   Reliability, Performance}
% \vskip .1mm
% \noindent {\small \bf Keywords:} {\small \mykeywords}
% 
% \end{sloppypar}

%%%%%%%%%%%%%%%%%%%%%%%%%%%%%%%%%%%
% Add page number.
\setcounter{page}{1}
\pagenumbering{arabic}

\thispagestyle{plain}
\pagestyle{plain}
\setlength{\footskip}{20pt}
%%%%%%%%%%%%%%%%%%%%%%%%%%%%%%%%%%%

% \begin{sloppypar}
% 
% \vskip 2mm
% \noindent {\small \bf Categories and Subject Descriptors:}
% {\small D.4.5~[{\bf Operating Systems}]: {Threads, Reliability};
%   C.2.4~[{\bf Computer-communication Networks}]: {Distributed Systems};}
% \vskip .1mm
% \noindent {\small \bf General Terms:} {\small Algorithms, Design,
%   Reliability, Performance}
% \vskip .1mm
% \noindent {\small \bf Keywords:} {\small \mykeywords}
% 
% \end{sloppypar}

\begin{sloppypar}
State machine replication (\smr) is a powerful fault-tolerance 
concept~\cite{paxos:practical}. It runs replicas 
of same program on multiple nodes. To keep the replicas consistent, it invokes a 
distributed consensus protocol (typically 
\paxos~\cite{paxos:simple,lamport98parttime,paxos:complex}) to ensure that a 
quorum (typically majority) of the replicas agree on the input request sequence. 
\smr is proven to tolerate various failure scenarios, like network 
partition and packet loss.

The fault-tolerant benefit of \smr makes it particularly an attractive 
high-availability service for general server programs. Unfortunately, despite 
much effort, existing \smr systems are still hard to deploy, mainly due to 
three problems.

\textbf{High consensus latency.} The consensus latency of traditional \paxos 
protocols is notoriously high, incurring high performance overhead for server 
programs. A main reason is that messages of traditional TCP or UDP-based \paxos 
protocols have to go through OS kernel. For efficiency, \paxos protocols 
typically take the Multi-Paxos approach~\cite{paxos:simple}: it assigns one 
replica as the ``leader'' to invoke consensus requests, and the other replicas 
as ``backups'' to agree on requests. To agree on an input, at least one 
round-trip time (RTT) is required between the leader and a backup. Given that a 
ping RTT in LAN typically takes hundreds of \us, and that the request processing 
time of key-value store servers (\eg, \redis) is at most hundreds of \us, 
existing \paxos protocols incur high overhead in the response time of server 
programs.

\textbf{Poor scalability.} The consensus latency of extant consensus protocols 
is often \emph{scale-limited}: it increases drastically when the number of 
concurrent requests or replicas 
increases~\cite{zookeeper,glendenning2011scalable}.

\begin{figure*}[ht]
\begin{center}
\includegraphics{figures/paxos_latency}
\caption{{\em Consensus latency of four existing consensus protocols.} All four 
protocols ran a client with 24 concurrent connections.}\label{fig:paxos_latency}
\end{center}
\end{figure*}

To quantify this problem, we evaluated four traditional consensus 
protocols~\cite{zookeeper, spaxos:srds12, crane:sosp15, libpaxos} on 24-core 
hosts with 40Gbps network. For each protocol, we spawned 24 concurrent 
consensus connections. As shown in Figure~\ref{fig:paxos_latency}, when 
changing the replica group size from 3 to 9, the consensus latency of three 
traditional protocols increased almost linearly to the number of replicas except 
S-Paxos. S-Paxos batches requests from replicas and invokes consensus when the 
batch is full. More replicas can take shorter time to form a batch, so S-Paxos 
incurred a slightly better consensus latency with more replicas. Nevertheless, 
its latency was always over 600\us.

To find scalability bottlenecks in traditional protocols, we used only one 
client connection and broke down their consensus latency on leader 
(Table~\ref{tab:traditional-latency}). From 3 to 9 replicas, the consensus 
latency (the ``Latency'' column) of these protocols increased more gently than 
that on 24 concurrent connections. For instance, when the number of replicas 
increased from three to nine, \zookeeper latency increased by 30.3\% with one 
connection; this latency increased by 168.3\% with 24 connections (Figure 1). 
This indicates that concurrent consensus requests are the major scalability 
bottleneck for these protocols.

\begin{table}[h]
\caption{{\em Performance breakdown of traditional protocols on leader
with only one connection.} The ``Proto-\#Rep" column is the protocol name 
and replica group size; ``Latency" is the consensus latency; ``First" 
is the latency of leader's first received consensus reply; ``Major" is the
latency of leader's consensus; ``Process" is leader's time spent in
processing all replies; and ``Sys" is leader's time spent in systems (OS
kernel, network stacks, and JVM) between the ``First" and ``Major" 
reply. Times are in \us.}
\label{tab:traditional-latency}
\centering
\begin{tabular}{lrrrrr}
\toprule
{\bf Proto-\#Rep} & {\bf Latency} & {\bf First} & {\bf Major} & {\bf
Process}
& {\bf Sys}\\
\midrule
\libpaxos-3 & 81.6 & 74.0  & 81.6 & 2.5 & 5.1\\
\libpaxos-9 & 208.3 & 145.0  & 208.3 & 12.0 & 51.3\\

\zookeeper-3 & 99.0 & 67.0  & 99.0 & 0.84 & 31.2\\
\zookeeper-9 & 129.0 & 76.0  & 128.0 & 3.6 & 49.4\\

\crane-3 & 78.0 & 69.0  & 69.0 & 13.0 & 0\\
\crane-9 & 148.0 & 83.0  & 142.0 & 30.0 & 35.0\\

\spaxos-3 & 865.1 & 846.0  & 846.0 & 20.0 & 0\\
\spaxos-9 & 739.1 & 545.0  & 731.0 & 35.0 & 159.1\\
\bottomrule
\end{tabular}
\end{table}

Specifically, three protocols had scalable latency on the arrival of their 
first consensus reply (the ``First'' column), which implies that network is not 
saturated. \libpaxos is an exception because its two-round protocol consumed 
much bandwidth. However, on the leader, there is a big gap between the arrival 
of the first consensus reply and the ``majority'' reply (the ``Major'' column). 
Given that the replies' CPU processing time was small (the ``Process'' column), 
we can see that various systems layers, including OS kernel, network 
libraries, and language runtimes (e.g., JVM), are another major scalable 
bottleneck (the ``Sys'' column).
 
This evaluation shows that both the number of concurrent requests and 
replicas make consensus latency increase drastically. As modern server programs 
tend to support more concurrent client connections, and advanced SMR systems 
tend to deploy more replicas (\eg, Azure~\cite{azure:book} deploys seven or 
nine replicas) to support both replica failures and upgrades, the limited 
scalability in extant consensus protocols becomes even more pronounced.

\textbf{Hard to use.} Most existing \smr systems are not designed to 
support unmodified server programs. To utilize existing \smr services, 
developers often have to rewrite their code to orchestrate server programs into 
the narrowly defined interfaces provided by these \smr systems. For instance, 
to leverage \zookeeper~\cite{zookeeper}, developers have to shoehorn their 
programs into the file IO interface defined by \zookeeper.

\end{sloppypar}

% uncomment to tweak with bib spacing
%\setlength\bibsep{2.25pt}
{
%\small
 \bibliographystyle{abbrv}
 \bibliography{bib/biblio}
}

\end{document}
